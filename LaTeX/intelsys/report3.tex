\documentclass[a4paper,11pt,dvipdfmx]{jsarticle}


% 数式
\usepackage{amsmath,amsfonts}
\usepackage{amsthm}
\usepackage{bm}

% 画像
\usepackage[dvipdfmx]{graphicx}

% 図形
\usepackage{tikz}
\usetikzlibrary{shapes.geometric}
\usetikzlibrary {shapes.misc}

% ソースコード
\usepackage{listings,jlisting,color}
\lstset{
basicstyle={\ttfamily},
identifierstyle={\small},
commentstyle={\smallitshape},
keywordstyle={\small\bfseries},
ndkeywordstyle={\small},
stringstyle={\small\ttfamily},
frame={tb},
breaklines=true,
columns=[l]{fullflexible},
numbers=left,
xrightmargin=0zw,
xleftmargin=3zw,
numberstyle={\scriptsize},
stepnumber=1,
numbersep=1zw,
lineskip=-0.5ex
}
\renewcommand{\lstlistingname}{ソースコード}

\renewcommand{\qedsymbol}{証明終了}



\begin{document}

\title{インテリジェントシステム レポート課題2}
\author{21T2166D 渡辺大樹}
\date{\today}
\maketitle

\section{}
成立するものは(a),(c),(d),(e)である。

誤っているのは(b),(f)となる。(b)は
$P(X|Y)=\frac{P(X,Y)}{\sum_{y}^{}P(X|Y=y)}$ではなく、$P(X|Y)=\frac{P(X,Y)}{P(Y)}$のため
$P(X|Y)=\frac{P(X,Y)}{\sum_{x}^{}P(X=x|Y)}$が正しい。

(f)は
$P(X_1,X_2,\cdots,X_n) = P(X_n)\prod_{i=1}^{n-1}P(X_i|X_n,\cdots,X_{n-i+1})$
が与えられている。$n=3$のときを考えてみると与式は$P(X_1,X_2,X_3)=P(X_3)P(X_1|X_3)P(X_2|X_1)$
となり、ベイズの定理から式変形すると$P(X_1,X_2,X_3)=\frac{P(X_1,X_3)P(X_2,X_1)}{P(X_1)}$
となる。この式は確率変数Xがそれぞれ独立のときのみ成り立つため、一般に成り立つとは言えない。

\section{}
(a)\[P(\lnot b)=0.45\]

(b)\[P(c)=0.30\]

(c)
\begin{align*}
    P(c|d) &= \frac{P(c,d)}{P(d)}\\
           &= \frac{0.3}{0.8}\\
           &= 0.37
\end{align*}

(d)
\begin{align*}
    P(a|d) &= \frac{P(a,d)}{P(d)}
    \intertext{ここで$P(a,C,d)=P(a)P(C|a)P(d|C)$より、}
    P(a,d) &= \sum_{C}P(a)P(C|a)P(d|C)\\
           &= P(a)P(c|a)P(d|c) + P(a)P(\lnot c|a)P(d|\lnot c)\\
           &= 0.298
    \intertext{したがって}
    P(a|d) &= 0.37
\end{align*}

\section{}
(c),(d)

\section{}
ややこしいのでノード集合を$\mathbb{C}$で表す。

(a)\(P(A,B)=P(A)P(B)\)を示す。
\begin{proof}[]
\begin{align*}
\intertext{$C \notin \mathbb{C}$より}
    P(A,B) &= \sum_{C}P(A,B,C)\\
           &= \sum_{C}\frac{P(A)P(B)P(C|A,B)}{P(C)}\\
           &= P(A)P(B) \sum_{C}\frac{P(C)}{P(C|A,B)}\\
\intertext{ここで$\sum_{C}\frac{P(C)}{P(C|A,B)}=1$より}
    P(A,B) &= P(A)P(B)
\end{align*}
\end{proof}

(b)条件付独立のため、\(P(B,C|A)=P(B|A)P(C|A)\)を示す。
\begin{proof}[]
    \begin{align*}
        \intertext{$A \in \mathbb{C}$より}
        P(A,B,C) &= P(A)P(B|A)P(C|A) \\
        P(B,C|A) &= \frac{P(A,B,C)}{P(A)} \\
        \intertext{$P(A,B,C) = P(A)P(B|A)P(C|A)$を代入して}
                 &= P(B|A)P(C|A)
    \end{align*}
\end{proof}

(c)条件付独立$P(A,C|B)=P(A|B)P(C|B)$を示す。
\begin{proof}[]
    \begin{align*}
        \intertext{$B \in \mathbb{C}$より}
        P(A,B,C) &= P(A)P(B|A)P(C|B) \\
        P(A,C|B) &= \frac{P(A)P(B|A)P(C|B)}{P(Y)}\\
        \intertext{ここで$P(A)P(B|A)=P(A,B)$より}
                 &= \frac{P(A,B)}{P(B)}P(C|B)\\
                 &= (A|B)P(C|B)
    \end{align*}
\end{proof}

\section{}
(a)$P(A,B,C,D,E,F,G)=P(A)P(B)P(C|A)P(D|A,B)P(E|D)P(F|D)P(G|C,E)$

(b)A,B,f

(c)G,E,f

\section{}
(a)\[P(X_1,X_2)\]



\end{document}