\documentclass[a4paper,11pt,dvipdfmx]{jsarticle}


% 数式
\usepackage{amsmath,amsfonts}
\usepackage{bm}

% 画像
\usepackage[dvipdfmx]{graphicx}

% 図形
\usepackage{tikz}
\usetikzlibrary{shapes.geometric}
\usetikzlibrary {shapes.misc}

% ソースコード
\usepackage{listings,jlisting,color}
\lstset{
basicstyle={\ttfamily},
identifierstyle={\small},
commentstyle={\smallitshape},
keywordstyle={\small\bfseries},
ndkeywordstyle={\small},
stringstyle={\small\ttfamily},
frame={tb},
breaklines=true,
columns=[l]{fullflexible},
numbers=left,
xrightmargin=0zw,
xleftmargin=3zw,
numberstyle={\scriptsize},
stepnumber=1,
numbersep=1zw,
lineskip=-0.5ex
}
\renewcommand{\lstlistingname}{ソースコード}


\begin{document}

\title{インテリジェントシステム レポート課題4}
\author{21T2166D 渡辺大樹}
\date{2023/07/21}
\maketitle

\section{}
(a)

1回Bellman updateを行った状態価値関数$U_1(s)$の一般式は
\[U_1(s)=\max_{a\in\{a_1,a_2\}}\sum_{s'}P(s'|s,a)[R(s,a,a')+\gamma U_0(s')]\]
となる。

今回初期値として与えられる$U_0$はすべて0なので
\[U_1(s)=\max_{a\in\{a_1,a_2\}}\sum_{s'}P(s'|s,a)R(s,a,a')\]
と書いてしまって計算する。またこの環境での$s$の取りうる値も$s_1,s_2$のどちらかであるため$U_1(s_0)$は0である。

まず$U_1(s_1)$を計算する。$U_1(s_1)$は状態行動価値関数$Q_1(s_1,a)=\sum_{s'}P(s'|s,a)R(s,a,a')$を用いて
\[U_1(s_1)=\max_{a\in\{a_1,a_2\}}Q_1(s_1,a)\]
と表せる。$Q(s,a)$は課題資料中の表から計算することで
\[Q_1(s_1,a_1)=1,Q_1(s_1,a_2)=2\]
となるため、二つから最大値を取って
\[U_1(s_1)=2\]
と計算できる。

$s_2$でも同様な計算を行うと
\[Q_1(s_2,a_1)=2,Q_1(s_2,a_2)=-10\]
となり、$U_1(s_2)$は
\[U_1(s_2)=2\]
となる。

表で示すと以下のようになる。
\begin{center}
    \begin{tabular}[h]{|c|c|c|c|} \hline
              & $s_1$ & $s_2$ & $s_0$ \\ \hline
        $U_1$ &   2   &   2   &   0   \\ \hline
    \end{tabular}
\end{center}

(b)

(a)からBellman updateをもう一度行うと状態価値関数$U_2(s)$の一般式は
\[U_2(s)=\max_{a\in\{a_1,a_2\}}\sum_{s'}P(s'|s,a)[R(s,a,a')+\gamma U_1(s')]\]
となる。また(a)と同様$U_2(s_0)=0$である。

まず$U_2(s_1)$を計算する。状態行動価値関数$Q_2(s,a)$は
\[Q_2(s,a)=\sum_{s'}P(s'|s,a)[R(s,a,a')+\gamma U_1(s')]\]
となる。課題資料中の表、また(a)の回答より$Q_2(s_1,a)$を$a$についてそれぞれ計算することで
\[Q_2(s_1,a_1)=2,Q_2(s_1,a_2)=3\]
が得られる。これの最大値を取るので
\[U_2(s_1)=3\]
となる。

同様にして$U_2(s_2)$も計算していくと
\[Q_2(s_2,a_1)=2,Q_2(s_2,a_2)=-9\]
となる。したがって最大値を取ることで
\[U_2(s_2)=2\]
を得る。

表で表すと以下のようになる。
\begin{center}
    \begin{tabular}[h]{|c|c|c|c|} \hline
              & $s_1$ & $s_2$ & $s_0$ \\ \hline
        $U_2$ &   3   &   2   &   0   \\ \hline
    \end{tabular}
\end{center}

(c)

方策評価によって得られる価値関数$U^{\pi_i}=U_i(s)$は
\[U_i(s)=\sum_{s'}P(s'|s,\pi_i(s))[R(s,\pi_i(s),s')+\gamma U_i(s')]\]
となる。今回求めるのは$U^{\pi_i}=U_0(s)$であるので、式は
\[U_0(s)=\sum_{s'}P(s'|s,\pi_0(s))R(s,\pi_0(s),s')\]
としてしまう。課題資料にある$\pi_0$を用いて$s_1$から計算していく。

$U_0(s_1)$は
\begin{align*}
U_0(s_1) & =P(s_1|s_1,a_1)R(s_1,a_1,s_1)\\
         & =1
\end{align*}
となる。同様に$U_0(s_2)=-10,U_0(s_0)=0$と計算できる。

表で表すと以下のようになる。
\begin{center}
    \begin{tabular}[h]{|c|c|c|c|} \hline
              & $s_1$ & $s_2$ & $s_0$ \\ \hline
        $U_0$ &   1   &  -10  &   0   \\ \hline
    \end{tabular}
\end{center}

(d)

方策$\pi_1(s)$は
\[\pi_1(s)=\arg \max_{a}\sum_{s'}P(s'|s,a)[R(s,a,s')+\gamma U_0(s')]\]
となる。$\gamma = \frac{1}{2}$で、$U_0(s)$は前問の値を使う。

状態行動価値関数$Q(s,a)$を$s=s_1,s_2$について$a_1,a_2$それぞれで計算すると
\begin{align*}
    Q(s_1,a_1) & =P(s_1|s_1,a_1)[R(s_1,a_1,s_1)+\frac{1}{2}U_0(s_1)] \\
               & =1.5\\
    Q(s_1,a_2) & =P(s_1|s_1,a_2)[R(s_1,a_2,s_1)+\frac{1}{2}U_0(s_1)] \\ 
               & \qquad + P(s_2|s_1,a_2)[R(s_1,a_2,s_2)+\frac{1}{2}U_0(s_2)]\\
               & =-0.25\\
    Q(s_2,a_1) & =P(s_1|s_2,a_1)[R(s_2,a_1,s_1)+\frac{1}{2}U_0(s_1)] \\
               & \qquad + P(s_2|s_2,a_1)[R(s_2,a_1,s_2)+\frac{1}{2}U_0(s_2)]\\
               & =-1.25\\
    Q(s_2,a_2) & =P(s_0|s_2,a_2)[R(s_2,a_2,s_0)+\frac{1}{2}U_0(s_0)] \\
               & =-10
\end{align*}

よって以上の結果から
\[\pi_1(s_1)=a_1,\pi_1(s_2)=a_1\]
となる。

表で表すと以下のようになる。
\begin{center}
    \begin{tabular}[h]{|c|c|c|} \hline
                & $s_1$ & $s_2$ \\ \hline
        $\pi_1$ & $a_1$ & $a_1$ \\ \hline
    \end{tabular}
\end{center}


\end{document}