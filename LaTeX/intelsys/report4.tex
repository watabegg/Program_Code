\documentclass[a4paper,11pt,dvipdfmx]{jsarticle}


% 数式
\usepackage{amsmath,amsfonts}
\usepackage{bm}

% 画像
\usepackage[dvipdfmx]{graphicx}

% 図形
\usepackage{tikz}
\usetikzlibrary{shapes.geometric}
\usetikzlibrary {shapes.misc}

% ソースコード
\usepackage{listings,jlisting,color}
\lstset{
basicstyle={\ttfamily},
identifierstyle={\small},
commentstyle={\smallitshape},
keywordstyle={\small\bfseries},
ndkeywordstyle={\small},
stringstyle={\small\ttfamily},
frame={tb},
breaklines=true,
columns=[l]{fullflexible},
numbers=left,
xrightmargin=0zw,
xleftmargin=3zw,
numberstyle={\scriptsize},
stepnumber=1,
numbersep=1zw,
lineskip=-0.5ex
}
\renewcommand{\lstlistingname}{ソースコード}


\begin{document}

\title{インテリジェントシステム レポート課題4}
\author{21T2166D 渡辺大樹}
\date{2023/07/21}
\maketitle

\section{}
(a)

1回Bellman updateを行った状態価値関数$U_1(s)$の一般式は
\[U_1(s)=\max_{a\in\{a_1,a_2\}}\sum_{s'}P(s'|s,a)[R(s,a,a')+\gamma U_0(s')]\]
となる。

今回初期値として与えられる$U_0$はすべて0なので
\[U_1(s)=\max_{a\in\{a_1,a_2\}}\sum_{s'}P(s'|s,a)R(s,a,a')\]
と書いてしまって計算する。またこの環境での$s$の取りうる値も$s_1,s_2$のどちらかであるため$U_1(s_0)$は0である。

まず$U_1(s_1)$を計算する。$U_1(s_1)$は状態行動価値関数$Q(s_1,a)=\sum_{s'}P(s'|s,a)R(s,a,a')$を用いて
\[U_1(s_1)=\max_{a\in\{a_1,a_2\}}Q(s_1,a)\]
と表せる。$Q(s,a)$は課題資料中の表から計算することで
\[Q(s_1,a_1)=1,Q(s_1,a_2)=2\]
となるため、二つの最大値を取って
\[U_1(s_1)=2\]
と計算できる。

$s_2$でも同様な計算を行うと
\[Q(s_2,a_1)=2,Q(s_2,a_2)=-10\]
となり、$U_1(s_2)$は
\[U_1(s_2)=2\]
となる。

\begin{center}
    \begin{tabular}[h]{|c|c|c|c|} \hline
              & $s_1$ & $s_2$ & $s_0$ \\ \hline
        $U_1$ &   2   &   2   &   0   \\ \hline
    \end{tabular}
\end{center}

(b)



\end{document}