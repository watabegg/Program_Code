\documentclass[a4paper,11pt,dvipdfmx]{jsarticle}


% 数式
\usepackage{amsmath,amsfonts}
\usepackage{bm}

% 画像
\usepackage[dvipdfmx]{graphicx}

% 図形
\usepackage{tikz}
\usetikzlibrary{shapes.geometric}
\usetikzlibrary {shapes.misc}

% ソースコード
\usepackage{listings,jlisting,color}
\lstset{
basicstyle={\ttfamily},
identifierstyle={\small},
commentstyle={\smallitshape},
keywordstyle={\small\bfseries},
ndkeywordstyle={\small},
stringstyle={\small\ttfamily},
frame={tb},
breaklines=true,
columns=[l]{fullflexible},
numbers=left,
xrightmargin=0zw,
xleftmargin=3zw,
numberstyle={\scriptsize},
stepnumber=1,
numbersep=1zw,
lineskip=-0.5ex
}
\renewcommand{\lstlistingname}{ソースコード}


\begin{document}

\title{インテリジェントシステム レポート課題4}
\author{21T2166D 渡辺大樹}
\date{2024/07/08}
\maketitle

\section{}
(a)
状態価値関数U(s)はここでは次の式で表すことができる。
\[U(s)=\max(R(s,stop), \sum_{s'}P(s'|s,spin)[R(s,spin,s')+\gamma U(s')])\]

状態s=0のときは
\[U(0)=\max(0, \sum_{s'}P(s'|0,spin)[R(0,spin,s')+\gamma U(s')])\]
となる。ここで
\[\sum_{s'}P(s'|0,spin)[R(0,spin,s')+\gamma U(s')]\]
を計算すると
\begin{align*}
    \sum_{s'}P(s'|0,spin)[R(0,spin,s')+\gamma U(s')] &= \frac{1}{3}(U_0(2)+U_0(3)+U_0(4)) \\
                                                     &= \frac{1}{3}(0+0+0) \\
                                                     &= 0
\end{align*}
となる。よって
\[U(0)=\max(0,0)=0\]
となる。

状態s=2のときは
\[U(2)=\max(2, \sum_{s'}P(s'|2,spin)[R(2,spin,s')+\gamma U(s')])\]
となる。ここで
\[\sum_{s'}P(s'|2,spin)[R(2,spin,s')+\gamma U(s')]\]
を計算すると
\begin{align*}
    \sum_{s'}P(s'|2,spin)[R(2,spin,s')+\gamma U(s')] &= \frac{1}{3}(U_0(4)+U_0(5)+0) \\
                                                     &= \frac{1}{3}(0+0+0) \\
                                                     &= 0
\end{align*}
となる。よって
\[U(2)=\max(2,0)=2\]

状態s=3のときは
\[U(3)=\max(3, \sum_{s'}P(s'|3,spin)[R(3,spin,s')+\gamma U(s')])\]
となる。ここで
\[\sum_{s'}P(s'|3,spin)[R(3,spin,s')+\gamma U(s')]\]
を計算すると
\begin{align*}
    \sum_{s'}P(s'|3,spin)[R(3,spin,s')+\gamma U(s')] &= \frac{1}{3}(U_0(5)+0+0) \\
                                                     &= \frac{1}{3}(0+0+0) \\
                                                     &= 0
\end{align*}
となる。よって
\[U(3)=\max(3,0)=3\]

状態s=4のときは
\[U(4)=\max(4, \sum_{s'}P(s'|4,spin)[R(4,spin,s')+\gamma U(s')])\]
となる。ここで
\[\sum_{s'}P(s'|4,spin)[R(4,spin,s')+\gamma U(s')]\]
を計算すると
\begin{align*}
    \sum_{s'}P(s'|4,spin)[R(4,spin,s')+\gamma U(s')] &= \frac{1}{3}(0+0+0) \\
                                                     &= \frac{1}{3}(0+0+0) \\
                                                     &= 0
\end{align*}
となる。よって
\[U(4)=\max(4,0)=4\]

状態s=5のときは
\[U(5)=\max(5, \sum_{s'}P(s'|5,spin)[R(5,spin,s')+\gamma U(s')])\]
となる。ここで
\[\sum_{s'}P(s'|5,spin)[R(5,spin,s')+\gamma U(s')]\]
を計算すると
\begin{align*}
    \sum_{s'}P(s'|5,spin)[R(5,spin,s')+\gamma U(s')] &= \frac{1}{3}(0+0+0) \\
                                                     &= \frac{1}{3}(0+0+0) \\
                                                     &= 0
\end{align*}
となる。よって
\[U(5)=\max(5,0)=5\]

(b)
\(U_2(s)\)を考える。
\(U_2(s)\)は次のように表される。
\[U_2(s)=\max(R(s,stop), \sum_{s'}P(s'|s,spin)[R(s,spin,s')+\gamma U_1(s')])\]

状態s=0のときは
\[U_2(0)=\max(0, \sum_{s'}P(s'|0,spin)[R(0,spin,s')+\gamma U_1(s')])\]
となる。ここで
\[\sum_{s'}P(s'|0,spin)[R(0,spin,s')+\gamma U_1(s')]\]
を計算すると
\begin{align*}
    \sum_{s'}P(s'|0,spin)[R(0,spin,s')+\gamma U_1(s')] &= \frac{1}{3}(U_1(2)+U_1(3)+U_1(4)) \\
                                                     &= \frac{1}{3}(2+3+4) \\
                                                     &= 3
\end{align*}
となる。よって
\[U_2(0)=\max(0,3)=3\]

状態s=2のときは
\[U_2(2)=\max(2, \sum_{s'}P(s'|2,spin)[R(2,spin,s')+\gamma U_1(s')])\]
となる。ここで
\[\sum_{s'}P(s'|2,spin)[R(2,spin,s')+\gamma U_1(s')]\]
を計算すると
\begin{align*}
    \sum_{s'}P(s'|2,spin)[R(2,spin,s')+\gamma U_1(s')] &= \frac{1}{3}(U_1(4)+U_1(5)+0) \\
                                                     &= \frac{1}{3}(4+5+0) \\
                                                     &= 3
\end{align*}
となる。よって
\[U_2(2)=\max(2,3)=3\]

状態s=3のときは
\[U_2(3)=\max(3, \sum_{s'}P(s'|3,spin)[R(3,spin,s')+\gamma U_1(s')])\]
となる。ここで
\[\sum_{s'}P(s'|3,spin)[R(3,spin,s')+\gamma U_1(s')]\]
を計算すると
\begin{align*}
    \sum_{s'}P(s'|3,spin)[R(3,spin,s')+\gamma U_1(s')] &= \frac{1}{3}(U_1(5)+0+0) \\
                                                     &= \frac{1}{3}(5+0+0) \\
                                                     &= \frac{5}{3}
\end{align*}
となる。よって
\[U_2(3)=\max(3,\frac{5}{3})=3\]

状態s=4のときは
\[U_2(4)=\max(4, \sum_{s'}P(s'|4,spin)[R(4,spin,s')+\gamma U_1(s')])\]
となる。ここで
\[\sum_{s'}P(s'|4,spin)[R(4,spin,s')+\gamma U_1(s')]\]
を計算すると
\begin{align*}
    \sum_{s'}P(s'|4,spin)[R(4,spin,s')+\gamma U_1(s')] &= \frac{1}{3}(0+0+0) \\
                                                     &= \frac{1}{3}(0+0+0) \\
                                                     &= 0
\end{align*}
となる。よって
\[U_2(4)=\max(4,0)=4\]

状態s=5のときは
\[U_2(5)=\max(5, \sum_{s'}P(s'|5,spin)[R(5,spin,s')+\gamma U_1(s')])\]
となる。ここで
\[\sum_{s'}P(s'|5,spin)[R(5,spin,s')+\gamma U_1(s')]\]
を計算すると
\begin{align*}
    \sum_{s'}P(s'|5,spin)[R(5,spin,s')+\gamma U_1(s')] &= \frac{1}{3}(0+0+0) \\
                                                     &= \frac{1}{3}(0+0+0) \\
                                                     &= 0
\end{align*}
となる。よって
\[U_2(5)=\max(5,0)=5\]

Bellman updateを繰り返し、同様に\(U_3(s),U_4(s)\)を求めると、
\begin{align*}
    U_3(0) &= 3 \\
    U_3(2) &= 3 \\
    U_3(3) &= 3 \\
    U_3(4) &= 4 \\
    U_3(5) &= 5 \\
    U_4(0) &= 3 \\
    U_4(2) &= 3 \\
    U_4(3) &= 3 \\
    U_4(4) &= 4 \\
    U_4(5) &= 5 \\
\end{align*}
となり、\(U(s)\)は
\begin{align*}
    U(0) &= 3 \\
    U(2) &= 3 \\
    U(3) &= 3 \\
    U(4) &= 4 \\
    U(5) &= 5
\end{align*}
に収束する。

(c)
\(U_4(s)\)を用いた最適な行動\(\pi(s) \in \{spin, stop\} \)を求める。

状態s=0のときは
\[\pi(0)=\arg\max_{\textbf{a}\in\{spin,stop\}}\sum_{s'}P(s'|0,\textbf{a})[R(0,\textbf{a},s')+\gamma U_4(s')]\]
となる。ここで
\[\sum_{s'}P(s'|0,spin)[R(0,spin,s')+\gamma U_4(s')]\]
を計算すると
\begin{align*}
    \sum_{s'}P(s'|0,spin)[R(0,spin,s')+\gamma U_4(s')] &= \frac{1}{3}(U_4(2)+U_4(3)+U_4(4)) \\
                                                     &= \frac{1}{3}(3+3+4) \\
                                                     &= \frac{10}{3}
\end{align*}
となる。
stopを選択したときは報酬が0なので
\[R(0,stop)=0\]
となる。よって
\[\pi(0)=\arg\max_{\textbf{a}\in\{spin,stop\}}(\frac{10}{3}, 0)\]
となり、\(\pi(0)=spin\)となる。

状態s=2のときは
\[\pi(2)=\arg\max_{\textbf{a}\in\{spin,stop\}}\sum_{s'}P(s'|2,\textbf{a})[R(2,\textbf{a},s')+\gamma U_4(s')]\]
となる。ここで
\[\sum_{s'}P(s'|2,spin)[R(2,spin,s')+\gamma U_4(s')]\]
を計算すると
\begin{align*}
    \sum_{s'}P(s'|2,spin)[R(2,spin,s')+\gamma U_4(s')] &= \frac{1}{3}(U_4(4)+U_4(5)+0) \\
                                                     &= \frac{1}{3}(4+5+0) \\
                                                     &= 3
\end{align*}
となる。
stopを選択したときは
\[R(2,stop)=2\]
となる。よって
\[\pi(2)=\arg\max_{\textbf{a}\in\{spin,stop\}}(3, 2)\]
となり、\(\pi(2)=spin\)となる。

状態s=3のときは
\[\pi(3)=\arg\max_{\textbf{a}\in\{spin,stop\}}\sum_{s'}P(s'|3,\textbf{a})[R(3,\textbf{a},s')+\gamma U_4(s')]\]
となる。ここで
\[\sum_{s'}P(s'|3,spin)[R(3,spin,s')+\gamma U_4(s')]\]
を計算すると
\begin{align*}
    \sum_{s'}P(s'|3,spin)[R(3,spin,s')+\gamma U_4(s')] &= \frac{1}{3}(U_4(5)+0+0) \\
                                                     &= \frac{1}{3}(5+0+0) \\
                                                     &= \frac{5}{3}
\end{align*}
となる。
stopを選択したときは
\[R(3,stop)=3\]
となる。よって
\[\pi(3)=\arg\max_{\textbf{a}\in\{spin,stop\}}(\frac{5}{3}, 3)\]
となり、\(\pi(3)=stop\)となる。

状態s=4のときは
\[\pi(4)=\arg\max_{\textbf{a}\in\{spin,stop\}}\sum_{s'}P(s'|4,\textbf{a})[R(4,\textbf{a},s')+\gamma U_4(s')]\]
となる。ここで
\[\sum_{s'}P(s'|4,spin)[R(4,spin,s')+\gamma U_4(s')]\]
を計算すると
\begin{align*}
    \sum_{s'}P(s'|4,spin)[R(4,spin,s')+\gamma U_4(s')] &= \frac{1}{3}(0+0+0) \\
                                                     &= \frac{1}{3}(0+0+0) \\
                                                     &= 0
\end{align*}
となる。
stopを選択したときは
\[R(4,stop)=4\]
となる。よって
\[\pi(4)=\arg\max_{\textbf{a}\in\{spin,stop\}}(0, 4)\]
となり、\(\pi(4)=stop\)となる。

状態s=5のときは
\[\pi(5)=\arg\max_{\textbf{a}\in\{spin,stop\}}\sum_{s'}P(s'|5,\textbf{a})[R(5,\textbf{a},s')+\gamma U_4(s')]\]
となる。ここで
\[\sum_{s'}P(s'|5,spin)[R(5,spin,s')+\gamma U_4(s')]\]
を計算すると
\begin{align*}
    \sum_{s'}P(s'|5,spin)[R(5,spin,s')+\gamma U_4(s')] &= \frac{1}{3}(0+0+0) \\
                                                     &= \frac{1}{3}(0+0+0) \\
                                                     &= 0
\end{align*}
となる。
stopを選択したときは
\[R(5,stop)=5\]
となる。よって
\[\pi(5)=\arg\max_{\textbf{a}\in\{spin,stop\}}(0, 5)\]
となり、\(\pi(5)=stop\)となる。

以上より、最適な行動は
\begin{align*}
    \pi(0) &= spin \\
    \pi(2) &= spin \\
    \pi(3) &= stop \\
    \pi(4) &= stop \\
    \pi(5) &= stop
\end{align*}
となる。

(d)
初期方策を以下の図で定めたときの価値関数\(U^{\pi_0}(s)\)を考える。
\begin{center}
    \begin{tabular}[h]{c|c|c|c|c|c}
        & s = 0 & s = 2 & s = 3 & s = 4 & s = 5 \\
        \hline
        $\pi_0$ & spin & stop & spin  & stop & spin
    \end{tabular}
\end{center}

この方策のもとで価値関数\(U^{\pi_0}(s)\)は次の線形方程式を解くことで求めることができる。
\begin{align*}
    U^{\pi_0}(0) &= \sum_{s'}P(s'|0,spin)[R(0,spin,s')+\gamma U^{\pi_0}(s')] \\
                 &= \frac{1}{3}(U^{\pi_0}(2)+U^{\pi_0}(3)+U^{\pi_0}(4)) \\
    U^{\pi_0}(2) &= 2 \\
    U^{\pi_0}(3) &= \sum_{s'}P(s'|3,spin)[R(3,spin,s')+\gamma U^{\pi_0}(s')] \\
                 &= \frac{1}{3}(U^{\pi_0}(5)+0+0) \\
    U^{\pi_0}(4) &= 4 \\
    U^{\pi_0}(5) &= \sum_{s'}P(s'|5,spin)[R(5,spin,s')+\gamma U^{\pi_0}(s')] \\
                 &= \frac{1}{3}(0+0+0) \\
                 &= 0
\end{align*}
となる。これを解くと
\begin{align*}
    U^{\pi_0}(0) &= \frac{1}{3}(2+0+4) = 2 \\
    U^{\pi_0}(2) &= 2 \\
    U^{\pi_0}(3) &= \frac{1}{3}(0+0+0) = 0 \\
    U^{\pi_0}(4) &= 4 \\
    U^{\pi_0}(5) &= 0
\end{align*}
となる。

(e)
続いて、\(U^{\pi_0}(s)\)を用いて方策改善を行う。
\[\pi_1(s)=\arg\max_{\textbf{a}\in\{spin,stop\}}\sum_{s'}P(s'|s,\textbf{a})[R(s,\textbf{a},s')+\gamma U^{\pi_0}(s')]\]
となる。これを計算すると
\begin{align*}
    \pi_1(0) &= \arg\max_{\textbf{a}\in\{spin,stop\}}\sum_{s'}P(s'|0,\textbf{a})[R(0,\textbf{a},s')+\gamma U^{\pi_0}(s')] \\
             &= \arg\max_{\textbf{a}\in\{spin,stop\}}(\frac{1}{3}(U^{\pi_0}(2)+U^{\pi_0}(3)+U^{\pi_0}(4)), 0) \\
             &= \arg\max_{\textbf{a}\in\{spin,stop\}}(\frac{1}{3}(2+0+4), 0) \\
                &= \arg\max_{\textbf{a}\in\{spin,stop\}}(\frac{6}{3}, 0) \\
                &= \arg\max_{\textbf{a}\in\{spin,stop\}}(2, 0) \\
                &= spin
\end{align*}
となる。同様に計算すると
\begin{align*}
    \pi_1(2) &= \arg\max_{\textbf{a}\in\{spin,stop\}}(\frac{1}{3}(U^{\pi_0}(4)+U^{\pi_0}(5)+0), 2) \\
             &= \arg\max_{\textbf{a}\in\{spin,stop\}}(\frac{1}{3}(4+0+0), 2) \\
                &= \arg\max_{\textbf{a}\in\{spin,stop\}}(\frac{4}{3}, 2) \\
                &= \arg\max_{\textbf{a}\in\{spin,stop\}}(1.\dot{3}, 2) \\
                &= stop
\end{align*}
\begin{align*}
    \pi_1(3) &= \arg\max_{\textbf{a}\in\{spin,stop\}}(\frac{1}{3}(U^{\pi_0}(5)+0+0), 3) \\
             &= \arg\max_{\textbf{a}\in\{spin,stop\}}(\frac{1}{3}(0+0+0), 3) \\
                &= \arg\max_{\textbf{a}\in\{spin,stop\}}(0, 3) \\
                &= stop
\end{align*}
\begin{align*}
    \pi_1(4) &= \arg\max_{\textbf{a}\in\{spin,stop\}}(\frac{1}{3}(0+0+0), 4) \\
             &= \arg\max_{\textbf{a}\in\{spin,stop\}}(0, 4) \\
                &= stop
\end{align*}
\begin{align*}
    \pi_1(5) &= \arg\max_{\textbf{a}\in\{spin,stop\}}(\frac{1}{3}(0+0+0), 5) \\
             &= \arg\max_{\textbf{a}\in\{spin,stop\}}(0, 5) \\
                &= stop
\end{align*}
となる。

以上より、方策改善によって得られた方策は
\begin{center}
    \begin{tabular}[h]{c|c|c|c|c|c}
        & s = 0 & s = 2 & s = 3 & s = 4 & s = 5 \\
        \hline
        $\pi_1$ & spin & stop & stop  & stop & stop
    \end{tabular}
\end{center}
となる。

\section{}
(a)

状態$s_2,s_4,s_5$はそれぞれ$s_4,s_5,s_0$にしか遷移しないため価値関数$U$はそれぞれ
\[U(s_2)=80,U(s_4)=90,U(s_5)=100\]
となる。

(b)

状態$s_3$での状態価値関数$U(s_3)$は
\begin{align*}
    U(s_3) & =\max_{\textbf{a}\in\{a,b\}}\sum_{s'}P(s'|s_3,\textbf{a})[R(s_3,\textbf{a},s')+\gamma U(s')] \\
           & =\sum_{s'}P(s'|s_3,b)[R(s_3,b,s')+\gamma U(s')] \\
           & =P(s_4|s_3,b)[R(s_3,b,s_4)+\gamma U(s_4)] \\
           &\qquad + P(s_5|s_3,b)[R(s_3,b,s_5)+\gamma U(s_5)] 
    \intertext{となる。$P(s_4|s_3,b)=p,P(s_5|s_3,b)=q$を代入し、ほかの値も課題資料中の表の値を用いると、}
    U(s_3) & =80p + 96q 
    \intertext{$p+q=1$より、pまたはqで整理すると}
           & =96-16p \\
           & =16q +80
\end{align*}
となる。

(c)

状態$s_1$での状態価値関数$U(s_1)$は
\begin{align*}
    U(s_1) & =\max_{\textbf{a}\in\{a,b\}}\sum_{s'}P(s'|s_1,\textbf{a})[R(s_1,\textbf{a},s')+\gamma U(s')] \\
           & =\max(P(s_3|s_1,a)[R(s_1,a,s_3)+\gamma U(s_3)] ,P(s_2|s_1,b)[R(s_1,b,s_2)+\gamma U(s_2)])
\end{align*}
である。ここに資料中の値と前問で出した答えを代入すると
\[U(s_1) =\max (86-16p,85)\]
となる。この関数の解は

\begin{equation*}
U(s_1) =
\begin{cases}
86-16p & (p <= \frac{1}{16})\\
85 & \text{otherwise}
\end{cases}
\end{equation*}

となる。

\end{document}