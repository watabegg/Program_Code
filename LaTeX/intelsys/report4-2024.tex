\documentclass[a4paper,11pt,dvipdfmx]{jsarticle}


% 数式
\usepackage{amsmath,amsfonts}
\usepackage{bm}

% 画像
\usepackage[dvipdfmx]{graphicx}

% 図形
\usepackage{tikz}
\usetikzlibrary{shapes.geometric}
\usetikzlibrary {shapes.misc}

% ソースコード
\usepackage{listings,jlisting,color}
\lstset{
basicstyle={\ttfamily},
identifierstyle={\small},
commentstyle={\smallitshape},
keywordstyle={\small\bfseries},
ndkeywordstyle={\small},
stringstyle={\small\ttfamily},
frame={tb},
breaklines=true,
columns=[l]{fullflexible},
numbers=left,
xrightmargin=0zw,
xleftmargin=3zw,
numberstyle={\scriptsize},
stepnumber=1,
numbersep=1zw,
lineskip=-0.5ex
}
\renewcommand{\lstlistingname}{ソースコード}


\begin{document}

\title{インテリジェントシステム レポート課題4}
\author{21T2166D 渡辺大樹}
\date{2024/07/08}
\maketitle

\section{}
(a)
状態価値関数U(s)はここでは次の式で表すことができる。
\[U(s)=\max(R(s,stop), \sum_{s'}P(s'|s,spin)[R(s,spin,s')+\gamma U(s')])\]

状態s=0のときは
\[U(0)=\max(0, \sum_{s'}P(s'|0,spin)[R(0,spin,s')+\gamma U(s')])\]

となる。ここで
\[\sum_{s'}P(s'|0,spin)[R(0,spin,s')+\gamma U(s')]\]
を計算すると
\begin{align*}
    \sum_{s'}P(s'|0,spin)[R(0,spin,s')+\gamma U(s')] &= \frac{1}{3}(U_0(2)+U_0(3)+U_0(4)) \\
                                                     &= \frac{1}{3}(0+0+0) \\
                                                     &= 0
\end{align*}
となる。よって
\[U(0)=\max(0,0)=0\]
となる。

状態s=2のときは
\[U(2)=\max(2, \sum_{s'}P(s'|2,spin)[R(2,spin,s')+\gamma U(s')])\]

となる。ここで
\[\sum_{s'}P(s'|2,spin)[R(2,spin,s')+\gamma U(s')]\]
を計算すると
\begin{align*}
    \sum_{s'}P(s'|2,spin)[R(2,spin,s')+\gamma U(s')] &= \frac{1}{3}(U_0(4)+U_0(5)+0) \\
                                                     &= \frac{1}{3}(0+0+0) \\
                                                     &= 0
\end{align*}
となる。よって
\[U(2)=\max(2,0)=2\]

状態s=3のときは
\[U(3)=\max(3, \sum_{s'}P(s'|3,spin)[R(3,spin,s')+\gamma U(s')])\]

となる。ここで
\[\sum_{s'}P(s'|3,spin)[R(3,spin,s')+\gamma U(s')]\]
を計算すると
\begin{align*}
    \sum_{s'}P(s'|3,spin)[R(3,spin,s')+\gamma U(s')] &= \frac{1}{3}(U_0(5)+0+0) \\
                                                     &= \frac{1}{3}(0+0+0) \\
                                                     &= 0
\end{align*}
となる。よって
\[U(3)=\max(3,0)=3\]

状態s=4のときは

(b)

(c)

(d)

(e)

\section{}
(a)

状態$s_2,s_4,s_5$はそれぞれ$s_4,s_5,s_0$にしか遷移しないため価値関数$U$はそれぞれ
\[U(s_2)=80,U(s_4)=90,U(s_5)=100\]
となる。

(b)

状態$s_3$での状態価値関数$U(s_3)$は
\begin{align*}
    U(s_3) & =\max_{\textbf{a}\in\{a,b\}}\sum_{s'}P(s'|s_3,\textbf{a})[R(s_3,\textbf{a},s')+\gamma U(s')] \\
           & =\sum_{s'}P(s'|s_3,b)[R(s_3,b,s')+\gamma U(s')] \\
           & =P(s_4|s_3,b)[R(s_3,b,s_4)+\gamma U(s_4)] \\
           &\qquad + P(s_5|s_3,b)[R(s_3,b,s_5)+\gamma U(s_5)] 
    \intertext{となる。$P(s_4|s_3,b)=p,P(s_5|s_3,b)=q$を代入し、ほかの値も課題資料中の表の値を用いると、}
    U(s_3) & =80p + 96q 
    \intertext{$p+q=1$より、pまたはqで整理すると}
           & =96-16p \\
           & =16q +80
\end{align*}
となる。

(c)

状態$s_1$での状態価値関数$U(s_1)$は
\begin{align*}
    U(s_1) & =\max_{\textbf{a}\in\{a,b\}}\sum_{s'}P(s'|s_1,\textbf{a})[R(s_1,\textbf{a},s')+\gamma U(s')] \\
           & =\max(P(s_3|s_1,a)[R(s_1,a,s_3)+\gamma U(s_3)] ,P(s_2|s_1,b)[R(s_1,b,s_2)+\gamma U(s_2)])
\end{align*}
である。ここに資料中の値と前問で出した答えを代入すると
\[U(s_1) =\max (86-16p,85)\]
となる。この関数の解は

\begin{equation*}
U(s_1) =
\begin{cases}
86-16p & (p <= \frac{1}{16})\\
85 & \text{otherwise}
\end{cases}
\end{equation*}

となる。

\end{document}