\documentclass[a4paper,11pt,dvipdfmx]{jsarticle}


% 数式
\usepackage{amsmath,amsfonts}
\usepackage{bm}

% 画像
\usepackage[dvipdfmx]{graphicx}

% 図形
\usepackage{tikz}
\usetikzlibrary{shapes.geometric}
\usetikzlibrary {shapes.misc}

% ソースコード
\usepackage{listings,jlisting,color}
\lstset{
basicstyle={\ttfamily},
identifierstyle={\small},
commentstyle={\smallitshape},
keywordstyle={\small\bfseries},
ndkeywordstyle={\small},
stringstyle={\small\ttfamily},
frame={tb},
breaklines=true,
columns=[l]{fullflexible},
numbers=left,
xrightmargin=0zw,
xleftmargin=3zw,
numberstyle={\scriptsize},
stepnumber=1,
numbersep=1zw,
lineskip=-0.5ex
}
\renewcommand{\lstlistingname}{ソースコード}


\begin{document}

\title{インテリジェントシステム レポート課題2}
\author{21T2166D 渡辺大樹}
\date{2023/06/14}
\maketitle
\subsubsection*{1. ミニマックス戦略}
ミニマックス戦略を用いた探索でのノードの値は
\begin{center}
\begin{description}
    \item[A:] 4
    \item[B:] 4
    \item[C:] 3
    \item[D:] 9
    \item[E:] 15t
    \item[G:] 3
    \item[H:] 12
    \item[J:] 2
    \item[M:] 2
    \item[P:] 11 
\end{description}
\end{center}
となる。

\subsubsection*{2. アルファ$\cdot$ベータ枝刈り}
アルファ$\cdot$ベータ枝刈りを用いたときに省略できるのは"J-R","H-P","M-T","M-U"の辺となる。

\subsubsection*{3. ノードの条件}
ノード値の条件はKが4以下、Qが9以下、Sが3以下となる。その他のノードR,T,U,W,Xは探索が省略されるので任意のノード値でよい。


\end{document}