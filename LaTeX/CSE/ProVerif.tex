\documentclass[a4paper,11pt]{jsarticle}

% 数式
\usepackage{amsmath,amsfonts}
\usepackage{bm}

% 画像
\usepackage[dvipdfmx]{graphicx}

% 図形
\usepackage{tikz}
\usetikzlibrary{shapes.geometric}
\usetikzlibrary {shapes.misc}

% ソースコード
\usepackage{listings,jlisting,color}
\lstset{
basicstyle={\ttfamily},
identifierstyle={\small},
commentstyle={\smallitshape},
keywordstyle={\small\bfseries},
ndkeywordstyle={\small},
stringstyle={\small\ttfamily},
frame={tb},
breaklines=true,
columns=[l]{fullflexible},
numbers=left,
xrightmargin=0zw,
xleftmargin=3zw,
numberstyle={\scriptsize},
stepnumber=1,
numbersep=1zw,
lineskip=-0.5ex
}
\renewcommand{\lstlistingname}{ソースコード}


\begin{document}

\title{暗号プロトコル設計演習 第一週例題}
\author{21T2166D 渡辺大樹}
\date{\today}
\maketitle

\section*{演習内容}
ProVerifを用いて設計したメッセージ認証とデジタル署名のプロトコルは以下のソースコード\ref{mac},\ref{sig}となった。
\lstinputlisting[caption=MAC.pv, label=mac]{C:/bin/proverif2.04/tests/MAC.pv} 
\lstinputlisting[caption=SIG.pv, label=sig]{C:/bin/proverif2.04/tests/SIG.pv}

コードの内容はソースコード内のコメントで記している。


\end{document}