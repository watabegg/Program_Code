\documentclass[a4paper,11pt]{jsarticle}


% 数式
\usepackage{amsmath,amsfonts}
\usepackage{bm}

% 画像
\usepackage[dvipdfmx]{graphicx}

% ソースコード
\usepackage{listings,jlisting,color}
\lstset{
language={c},
backgroundcolor={\color[gray]{.85}},
basicstyle={\ttfamily},
identifierstyle={\small},
commentstyle={\small \color[rgb]{0,0.5,0}},
keywordstyle={\small\bfseries \color[rgb]{0.5,0,0.5}},
ndkeywordstyle={\small},
stringstyle={\small\ttfamily \color[rgb]{0,0,1}},
frame={tb},
breaklines=true,
columns=[l]{fullflexible},
numbers=left,
xrightmargin=0zw,
xleftmargin=3zw,
numberstyle={\scriptsize},
stepnumber=1,
numbersep=1zw,
lineskip=-0.5ex,
}
\renewcommand{\lstlistingname}{ソースコード}


\begin{document}

\title{プログラミング言語I\hspace{-1.2pt}I レポート課題9}
\author{21T2166D 渡辺大樹}
\date{\today}
\maketitle

レポート課題9では示された二つの構造体を用いたコードを比較していく。
まず比較する二つのコードをソースコード\ref{91},\ref{92}に示す。
\lstinputlisting[language=c, caption={code1}, label=91]{C:/Program_Code/c/ProgLang/Class09/report9_2.c}
\lstinputlisting[language=c, caption={code2}, label=92]{C:/Program_Code/c/ProgLang/Class09/report9.c}

この二つのコードは


\end{document}