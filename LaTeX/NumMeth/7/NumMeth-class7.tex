\documentclass[a4paper,11pt,dvipdfmx]{jsarticle}


% 数式
\usepackage{amsmath,amsfonts}
\usepackage{bm}

% 画像
\usepackage[dvipdfmx]{graphicx}
\usepackage{framed}

% 図形
\usepackage{tikz}
\usetikzlibrary{shapes.geometric}
\usetikzlibrary {shapes.misc}

% ソースコード
\usepackage{listings,jlisting,color}
\lstset{
basicstyle={\ttfamily},
identifierstyle={\small},
commentstyle={\smallitshape},
keywordstyle={\small\bfseries},
ndkeywordstyle={\small},
stringstyle={\small\ttfamily},
frame={tb},
breaklines=true,
columns=[l]{fullflexible},
numbers=left,
xrightmargin=0zw,
xleftmargin=3zw,
numberstyle={\scriptsize},
stepnumber=1,
numbersep=1zw,
lineskip=-0.5ex
}
\renewcommand{\lstlistingname}{ソースコード}


\begin{document}
\definecolor{shadecolor}{gray}{0.70}

\title{数値計算 Class-7 演習}
\author{21T2166D 渡辺大樹}
\date{2023/07/23}
\maketitle

\section{演習内容}
Class-7ではいくつかの$xy$平面上のデータからそのデータをすべて通るなめらかな曲線を描くための手法、
スプライン関数についてコードを実装し、実際に動かしていく。

スプライン関数はn個あるデータの隣り合うデータからデータの区間にそれぞれ同じ次数の多項式を定義しつなぎ合わせた、
区分的多項式である。区間の多項式の次数がnであるとき、n次のスプライン関数という。
この関数はその定義にあるようになめらかな曲線である必要がある。

関数がなめらかであるというのはその区間の両端で関数が連続かつ微分可能である必要がある。
n次のスプライン関数では$n-1$次の微分係数が区間の両端で隣接する区間と一致する必要がある。

以下では3次のスプライン関数$S(x)$を考えていく。

$xy$平面上に$n+1$個のデータ$(x_0,y_0),\cdots,(x_n,y_n)$(ただし$x_{i-1}<x_i,\quad(i=1,2,\cdots,n)$)が与えられてる。
この隣り合うデータがつくる$n$個の小区間$l_i=[x_{i-1},x_i]$上の3次関数$S_i(x)$を次の二つの条件を満たすように設定する。
\begin{enumerate}
    \item $S_i(x)$は区間の両端で連続、すなわちデータ点を通る。
    \begin{align*}
        S_i(x_{i-1}) & = y_{i-1}\\
        S_i(x_{i}) & = y_{i}
    \end{align*}
    \item 隣り合う関数がその節点となるデータ点で2階微分までの係数が一致する。
    \begin{align*}
        S'_i(x_i) & =S'_{i+1}(x_i) \\
        S''_i(x_i) & =S''_{i+1}(x_i)
    \end{align*}
\end{enumerate}

この$S_i(x)$をつないで区間$[x_0,x_n]$の関数にすることで、3次のスプライン関数$S(x)$が得られる。

この定義に合うよう、区間$l_i$上のスプライン関数$S_i(x)$を決めると
\[S_i(x)=A_i(x-x{i-1})^3+B_i(x-x_{i-1})^2+C_i(x-x_{i-1})+y_{i-1}\]
となる。$A_i,B_i,C_i$はそれぞれ定数である。

この各定数を$i$ごとに求める必要があり、それは以下の手法で求めることができる。

まず、簡略化のため
\[h_i=x_i-x_{x-1},\quad u_i=\frac{y_i-y_{i-1}}{h_i}\]
と置く。

一次の係数$C_i$は$x_0,x_n$における節点の傾き$C_1,C_{n+1}$を適当な値に定め、
\[h_iC_{i-1}+2(h_{i-1}+h_j)C_j+h_{i-1}C_{i+1}=3(h_iu_{i-1}+h_{i-1}u_i)\]
という$C_i$に関する連立方程式を解くことで求められる。

これを求めてしまえば残りの係数$B_i,A_i$は以下のように求められる。
\begin{align*}
    B_i &= \frac{3u_i-2C_i-C_{i+1}}{h_i}\\
    A_i &= \frac{u_i-h_iB_i-C_i}{h_i^2}
\end{align*}

これらを求めることで区間$l_i$のスプライン関数$S_i(x)$が求まり、つなげれば$S(x)$となる。

以上のスプライン関数を求めるアルゴリズムは以下のソースコード\ref{spline}で実装される。
\lstinputlisting[caption=spline.c, label=spline]{C:/Program_Code/NumMeth/Class7/spline.c}

コード内では必要な定数を入力させたあと、44行目から実際にスプライン関数の係数を計算している。

59行目からは1次の係数を求める連立方程式を解くため、
その係数をp[N+2][N+3]の係数行列におさめ、70行目からのガウスジョルダン法に用いている。

この連立方程式を解いた後は96行目以降で順次2次3次の係数も先ほどの計算方法を実装して解いている。

出力は二通りで、各区間のスプライン関数と、
そのスプライン関数を入力されたkz分して得られた補間点を出力できるように実装されている。

\section{演習結果}
実際に例題にある問題を解いた。

入力する補間点は
\begin{center}
    \begin{tabular}[h]{|c|c|c|c|c|c|} \hline
         $x$  & 0.0 & 1.0 & 1.5 & 2.0 & 3.0 \\ \hline
         $y$  & 2.0 & 4.0 & 3.0 & 1.0 & 2.0 \\ \hline
    \end{tabular}
\end{center}
である。

左端点における1次の微分係数は5,右端点における1次の微分係数は3とし、12等分した補間値を出力するように入力する。

結果として、出力されたスプライン関数は
\begin{shaded}
    \noindent
    \begin{align*}
        S_1(x) &= ( 0.383)(x- 0.000)^3+(-3.383)(x- 0.000)^2+( 5.000)(x- 0.000)+( 2.000)\\
        S_2(x) &= (-1.067)(x- 1.000)^3+(-2.233)(x- 1.000)^2+(-0.617)(x- 1.000)+( 4.000)\\
        S_3(x) &= ( 6.267)(x- 1.500)^3+(-3.833)(x- 1.500)^2+(-3.650)(x- 1.500)+( 3.000)\\
        S_4(x) &= (-1.783)(x- 2.000)^3+( 5.567)(x- 2.000)^2+(-2.783)(x- 2.000)+( 1.000)
    \end{align*}
\end{shaded}
という結果となった。

12等分した補間値は
\begin{shaded}
    \noindent
    0.000000   2.000000\\
    0.250000   3.044531\\
    0.500000   3.702083\\
    0.750000   4.008594\\
    1.000000   4.000000\\
    1.250000   3.689583\\
    1.500000   3.000000\\
    1.750000   1.945833\\
    2.000000   1.000000\\
    2.250000   0.750000\\
    2.500000   1.783333\\
    2.750000   4.687500\\
    3.000000  10.050000
\end{shaded}
となった。

補間値とスプライン関数をグラフに起こそうと思ったがまたの機会にしておく。

\end{document}