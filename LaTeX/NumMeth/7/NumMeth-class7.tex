\documentclass[a4paper,11pt,dvipdfmx]{jsarticle}


% 数式
\usepackage{amsmath,amsfonts}
\usepackage{bm}

% 画像
\usepackage[dvipdfmx]{graphicx}
\usepackage{framed}

% 図形
\usepackage{tikz}
\usetikzlibrary{shapes.geometric}
\usetikzlibrary {shapes.misc}

% ソースコード
\usepackage{listings,jlisting,color}
\lstset{
basicstyle={\ttfamily},
identifierstyle={\small},
commentstyle={\smallitshape},
keywordstyle={\small\bfseries},
ndkeywordstyle={\small},
stringstyle={\small\ttfamily},
frame={tb},
breaklines=true,
columns=[l]{fullflexible},
numbers=left,
xrightmargin=0zw,
xleftmargin=3zw,
numberstyle={\scriptsize},
stepnumber=1,
numbersep=1zw,
lineskip=-0.5ex
}
\renewcommand{\lstlistingname}{ソースコード}


\begin{document}
\definecolor{shadecolor}{gray}{0.70}

\title{数値計算 Class-7 演習}
\author{21T2166D 渡辺大樹}
\date{2023/07/23}
\maketitle

\section{演習内容}
Class-7ではいくつかの$xy$平面上のデータからそのデータをすべて通るなめらかな曲線を描くための手法、
スプライン関数についてコードを実装し、実際に動かしていく。

スプライン関数はn個あるデータの隣り合うデータからデータの区間にそれぞれ同じ次数の多項式を定義しつなぎ合わせた、
区分的多項式である。区間の多項式の次数がnであるとき、n次のスプライン関数という。
この関数はその定義にあるようになめらかな曲線である必要がある。

関数がなめらかであるというのはその区間の両端で関数が連続かつ微分可能である必要がある。
n次のスプライン関数では$n-1$次の微分係数が区間の両端で隣接する区間と一致する必要がある。

以下では3次のスプライン関数$S(x)$を考えていく。

$xy$平面上に$n+1$個のデータ$(x_0,y_0),\cdots,(x_n,y_n)$(ただし$x_{i-1}<x_i,\quad(i=1,2,\cdots,n)$)が与えられてる。
この隣り合うデータがつくる$n$個の小区間$l_i=[x_{i-1},x_i]$上の3次関数$S_i(x)$を次の二つの条件を満たすように設定する。
\begin{enumerate}
    \item $S_i(x)$は区間の両端で連続、すなわちデータ点を通る。
    \begin{align*}
        S_i(x_{i-1}) & = y_{i-1}\\
        S_i(x_{i}) & = y_{i}
    \end{align*}
    \item 隣り合う関数がその節点となるデータ点で2階微分までの係数が一致する。
    \begin{align*}
        S'_i(x_i) & =S'_{i+1}(x_i) \\
        S''_i(x_i) & =S''_{i+1}(x_i)
    \end{align*}
\end{enumerate}

この$S_i(x)$をつないで区間$[x_0,x_n]$の関数にすることで、3次のスプライン関数$S(x)$が得られる。

スプライン関数を求めるアルゴリズムは以下のソースコード\ref{spline}で実装される。
\lstinputlisting[caption=spline.c, label=spline]{C:/Program_Code/NumMeth/Class7/spline.c}



\end{document}