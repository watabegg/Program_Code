\documentclass[a4paper,11pt]{jsarticle}


% 数式
\usepackage{amsmath,amsfonts}
\usepackage{bm}

% 画像
\usepackage[dvipdfmx]{graphicx}

% ソースコード
\usepackage{listings,jlisting,color}
\lstset{
basicstyle={\ttfamily},
identifierstyle={\small},
commentstyle={\smallitshape},
keywordstyle={\small\bfseries},
ndkeywordstyle={\small},
stringstyle={\small\ttfamily},
frame={tb},
breaklines=true,
columns=[l]{fullflexible},
numbers=left,
xrightmargin=0zw,
xleftmargin=3zw,
numberstyle={\scriptsize},
stepnumber=1,
numbersep=1zw,
lineskip=-0.5ex
}
\renewcommand{\lstlistingname}{ソースコード}


\begin{document}

\title{数値計算 Class-5 演習}
\author{21T2166D 渡辺大樹}
\date{2023月5月10日}
\maketitle

\section{演習内容}
Class-5ではClass-2で演習したニュートン法を非線形の連立方程式の解決のために拡張し、実際に連立非線形方程式の解を見つけていく。

非線形の連立方程式を解くためには前回演習したLU分解による連立方程式の解法を用いたいのでライブラリとして
ソースコード\ref{v3}を読み込んでいく。
\lstinputlisting[caption={my\_library\_v3.h}, label=v3]{C:/Program_code/c/NumMeth5/my_library_v3.h}

ソースコード\ref{v3}にはベクトルのノルムを求めるためのvector\_norm1()やLU分解で連立一次方程式を解くlu\_solve()が実装されている。

$A\textbf{x}=\textbf{b}$について解くとき、lu\_solve()では前回実装したludecomp()関数で行列$A$をLU分解したのち、
$L\textbf{y}=\textbf{b}$と$U\textbf{x}=\textbf{y}$についてそれぞれを前進代入、後進代入で解いている。

では以降では実際に非線形連立方程式を解く段階を説明していく。

アルゴリズムは以下ソースコード\ref{non}に実装されている。
\lstinputlisting[caption={nonlinear\_system.c}, label=non]{C:/Program_code/c/NumMeth5/nonlinear_system.c}

\newpage
このソースコード\ref{non}の中で実装されている拡張されたニュートン法についてコードを交えながら説明する。

まず、以下の一意解を持つn元連立非線形方程式
\begin{displaymath}
    \begin{array}{lll}
        f_1(x_1,x_2,\cdots,x_n) &=& 0 \\
        f_2(x_1,x_2,\cdots,x_n) &=& 0 \\
        & \vdots & \\
        f_n(x_1,x_2,\cdots,x_n) &=& 0 
    \end{array}
\end{displaymath}
を考える。

\begin{displaymath}
    \begin{array}{lll}
        x &=& (x_1,x_2,\cdots,x_n) \\
        \textbf{f}(x) &=& [f_1(x), f_2(x), \cdots, f_n(x)]^t
    \end{array}
\end{displaymath}
とすると、
\begin{displaymath}
    \textbf{f}(x) = 0
\end{displaymath}
を求めればよい。

この式は非線形ではあるが一意の解を持つ連立方程式であるから、線形の連立方程式で用いたニュートン法を非線形に拡張し用いる。

線形一次の連立方程式では
\begin{equation*}
    x_{n+1} = x_n - \frac{f(x_n)}{f'(x_n)}
\end{equation*}
の数列が解に収束するまで繰り返していた。これは解を$\alpha$とすると$x=\alpha$についてのテイラー展開
を元に導出した式である。

これを非線形でもこの考え方を応用し、解$\alpha$の近くの点$x=x_{\alpha}$での$f_k(x)$のテイラー展開をすると
\begin{displaymath}
    f_k(\alpha) \approx f_k(x_\alpha) + \sum_{i=1}^{n}\frac{\partial f_k}{\partial x_i}(x_\alpha)(\alpha_i - {x_\alpha}_i)
\end{displaymath}
の式が得られる。この式がゼロになる点を解けばよい。

ここでこれを$\textbf{f}(x)$にまで拡張させると上記の式は
\begin{equation*}
    \begin{bmatrix}
        f_1(x_\alpha) \\ f_2(x_\alpha) \\ \vdots \\ f_n(x_\alpha)
    \end{bmatrix}
    + 
    \begin{bmatrix}
        \frac{\partial f_1}{\partial x_1}(x_\alpha) && \frac{\partial f_1}{\partial x_2}(x_\alpha) && \cdots && \frac{\partial f_1}{\partial x_n}(x_\alpha) \\
        \frac{\partial f_2}{\partial x_1}(x_\alpha) && \frac{\partial f_2}{\partial x_2}(x_\alpha) && \cdots && \frac{\partial f_2}{\partial x_n}(x_\alpha) \\
        \vdots && \vdots && \ddots && \vdots \\
        \frac{\partial f_n}{\partial x_1}(x_\alpha) && \frac{\partial f_n}{\partial x_2}(x_\alpha) && \cdots && \frac{\partial f_n}{\partial x_n}(x_\alpha) 
    \end{bmatrix}
    \begin{bmatrix}
        \alpha_1 - {x_\alpha}_1 \\ \alpha_2 - {x_\alpha}_2 \\ \vdots \\ \alpha_n - {x_\alpha}_n
    \end{bmatrix}
    =
    \begin{bmatrix}
        0 \\ 0 \\ \vdots \\ 0
    \end{bmatrix}
\end{equation*}
となる。

ここで$\alpha - x_\alpha$の係数行列は$x_\alpha$におけるヤコビアンであり、
\begin{equation*}
    J(x_\alpha) = 
    \begin{bmatrix}
        \frac{\partial f_1}{\partial x_1}(x_\alpha) && \frac{\partial f_1}{\partial x_2}(x_\alpha) && \cdots && \frac{\partial f_1}{\partial x_n}(x_\alpha) \\
        \frac{\partial f_2}{\partial x_1}(x_\alpha) && \frac{\partial f_2}{\partial x_2}(x_\alpha) && \cdots && \frac{\partial f_2}{\partial x_n}(x_\alpha) \\
        \vdots && \vdots && \ddots && \vdots \\
        \frac{\partial f_n}{\partial x_1}(x_\alpha) && \frac{\partial f_n}{\partial x_2}(x_\alpha) && \cdots && \frac{\partial f_n}{\partial x_n}(x_\alpha) 
    \end{bmatrix}
\end{equation*}
で表す。

$\alpha - x_\alpha$を$\Delta x$で表すとこの式は
\begin{equation*}
    \textbf{f}(x_\alpha) + J(x_\alpha) \Delta x = 0
\end{equation*}
となる。

この式を$\Delta x$について解けば
\begin{equation*}
    \Delta x = -[J(x_\alpha)]^{-1} \textbf{f}(x_\alpha)
\end{equation*}
となり、


\end{document}