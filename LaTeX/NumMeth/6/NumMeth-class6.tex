\documentclass[a4paper,11pt,dvipdfmx]{jsarticle}


% 数式
\usepackage{amsmath,amsfonts}
\usepackage{bm}

% 画像
\usepackage[dvipdfmx]{graphicx}

% 図形
\usepackage{tikz}
\usetikzlibrary{shapes.geometric}
\usetikzlibrary {shapes.misc}

% ソースコード
\usepackage{listings,jlisting,color}
\lstset{
basicstyle={\ttfamily},
identifierstyle={\small},
commentstyle={\smallitshape},
keywordstyle={\small\bfseries},
ndkeywordstyle={\small},
stringstyle={\small\ttfamily},
frame={tb},
breaklines=true,
columns=[l]{fullflexible},
numbers=left,
xrightmargin=0zw,
xleftmargin=3zw,
numberstyle={\scriptsize},
stepnumber=1,
numbersep=1zw,
lineskip=-0.5ex
}
\renewcommand{\lstlistingname}{ソースコード}


\begin{document}

\title{数値計算 Class-6 演習}
\author{21T2166D 渡辺大樹}
\date{2023/06/26}
\maketitle

\section{演習内容}
Class-6では伴って変化する二つの変数x,yについてその変化の関係を調べるため、
実験、観測などで得たいくつかのx,yの値を元にしてx,yの関係を推定する補間法についてCで実装する。

今回扱う補間法はラグランジュ法とニュートン法で、この2つについて以下に示していく。
\subsubsection{ラグランジュの補間法}
ラグランジュの補完法はソースコード\ref{lag}で実装される。
\lstinputlisting[caption=laghkn.c, label=lag]{C:/Program_Code/NumMeth/Class6/laghkn.c}

このコードの動作をラグランジュの補間法とともに解説していく。

今$x=x_1$のとき$y=y_1$、$x=x_2$のとき$y=y_2$、$x=x_3$のとき$y=y_3$であるような
定数$x_1,x_2,x_3,y_1,y_2,y_3$を考える。

まず定数を2つに絞って考える。
$x=x_1$のとき$y=y_1$でありたいのでxについての一次式$y=\frac{x-x_2}{x_1-x_2}y_1$のような式を考えると
$x=x_1$のとき$y=y_1$、$x=x_2$のとき$y=0$となる。同じように
$x=x_2$のとき$y=y_2$になるようなxの一次式を考えると$y=\frac{x-x_1}{x_2-x_1}y_2$となり、
$x=x_1$のとき$y=0$、$x=x_2$のとき$y=y_2$となる。

この二式を足し合わせることで
\begin{equation*}
    y=\frac{x-x_2}{x_1-x_2}y_1 + \frac{x-x_1}{x_2-x_1}y_2
\end{equation*}
という式が求まる。この式は2点$(x_1,y_1),(x_2,y_2)$を通る一次の直線を表している。

ここから3点$x_1,x_2,x_3,y_1,y_2,y_3$について考えていく。
$x=x_1$のとき$y=y_1$でありたいのでxについての一次式$y=\frac{(x-x_2)(x-x_3)}{(x_1-x_2)(x_1-x_3)}y_1$のような式を考えると
$x=x_1$のとき$y=y_1$、$x=x_2,x_3$のとき$y=0$となる。同様に考えると3式の和を考えて
\begin{equation*}
    y=\frac{(x-x_2)(x-x_3)}{(x_1-x_2)(x_1-x_3)}y_1 + \frac{(x-x_3)(x-x_1)}{(x_2-x_3)(x_2-x_1)}y_2 + \frac{(x-x_1)(x-x_2)}{(x_3-x_1)(x_3-x_2)}y_3
\end{equation*}
の式を得られる。これは$x=x_1$のとき$y=y_1$、$x=x_2$のとき$y=y_2$、$x=x_3$のとき$y=y_3$になっている。

したがってこの式は3点$(x_1,y_1),(x_2,y_2),(x_3,y_3)$を通る高々2次式を表す。

この式はデータの個数が増えても同様な式で表すことができて、
データの個数をn+1、点の組み合わせを$(x_j,y_j),(j=0,1,2,\cdots,n)$で表す。
まず$y_k$の係数となる部分を$L_k(x)$とすると
\begin{equation*}
    L_k(x) = \prod_{\substack{j=0 \\ j \neq k}}^{n} \frac{x-x_j}{x_k-x_j} (k=0,1,\cdots,n)
\end{equation*}
と表すことができる。

これを用いると$(x_j,y_j),(j=0,1,2,\cdots,n)$を通るn次の多項式$L(x)$は
\begin{equation*}
    L(x) = \sum_{k=0}^{n} L_k(x) \cdot y_k
\end{equation*}
と表せる。

これがラグランジュの補間多項式となる。

この処理が実際にソースコード\ref{lag}に実装されている。
具体的にこの計算が実装されているのは40-53行目であり、このfor文の中が$L(k)$の足し合わせの計算になっており
45行目からのfor文が$L_k(x)$の掛け合わせの計算になっている。

このコードでは、多項式を出力するのではなくデータの最大値と最小値の間で、
予測された多項式の出力したい点の個数を出力出来るようにしている。

\subsection{ニュートンの補間法}
ニュートンの補完法はソースコード\ref{newton}で実装される。
\lstinputlisting[caption=newton.c, label=newton]{C:/Program_Code/NumMeth/Class6/newhkn.c}




\end{document}