\documentclass[a4paper,11pt,titlepage,dvipdfmx]{jsarticle}


% 数式
\usepackage{amsmath,amsfonts}
\usepackage{bm}

% 画像
\usepackage[dvipdfmx]{graphicx}
\usepackage{framed}

% 図形
\usepackage{tikz}
\usetikzlibrary{shapes.geometric}
\usetikzlibrary {shapes.misc}

% ソースコード
\usepackage{listings,jlisting,color}
\lstset{
basicstyle={\ttfamily},
identifierstyle={\small},
commentstyle={\smallitshape},
keywordstyle={\small\bfseries},
ndkeywordstyle={\small},
stringstyle={\small\ttfamily},
frame={tb},
breaklines=true,
columns=[l]{fullflexible},
numbers=left,
xrightmargin=0zw,
xleftmargin=3zw,
numberstyle={\scriptsize},
stepnumber=1,
numbersep=1zw,
lineskip=-0.5ex
}
\renewcommand{\lstlistingname}{ソースコード}
\definecolor{shadecolor}{gray}{0.70}

\begin{document}
\begin{titlepage}
\noindent
\vspace{6cm}
\begin{center}
\begin{LARGE}
コンピュータアーキテクチャ \\
第1回課題 
\end{LARGE}
\end{center}
\vspace{6cm}
\begin{flushright}
信州大学工学部 \
電子情報システム工学科 \
\begin{description}
\setlength{\leftskip}{8.9cm}
\item 2024/10/04
\item 21T2166D 渡辺 大樹
\end{description}
\end{flushright}
\end{titlepage}


\section*{課題1}
\subsection*{1.プログラムはハードウェア上でどのように実行されるのか}
「プログラムはハードウェア上でどのように実行されるのか」についてAI(Chat GPT)を用い、「プログラムはハードウェア上でどのように実行されるのかを500字程度で説明してください」というプロンプトを実行し調べた結果を以下に示す。\\
\begin{shaded}
    プログラムは、ハードウェア上で次のように実行されます。\\
    まず、プログラムのコードは人間が読める高水準言語で書かれますが、そのままではハードウェアは理解できないため、コンパイラやインタプリタを使ってマシン語(0と1のバイナリ)に変換されます。\\
    このバイナリ形式のプログラムは、メモリ(RAM)にロードされます。\\
    その後、CPU(中央処理装置)がメモリから命令を読み取り、プログラムカウンタによって次に実行する命令を順次処理します。\\
    CPUの内部にはレジスタがあり、計算結果やメモリのアドレスが一時的に格納されます。\\
    また、算術演算ユニット(ALU)が加減乗除などの演算を行い、制御ユニットが命令の順序を管理します。\\
    入力や出力が必要な場合、プログラムはハードディスク、キーボード、ディスプレイなどの外部デバイスと通信し、デバイスドライバを介してハードウェアにアクセスします。\\
    すべての命令が実行されるまで、このプロセスが繰り返され、最終的にプログラムが完了します。\\
\end{shaded}

\subsection*{2.この結果においてわからないことをまとめる}
上記の結果においてわからないことを以下にまとめる。
\begin{itemize}
    \item プログラムカウンタとは何か
    \item ALUとは何か
    \item 制御ユニットとは何か、どういった制御を行うのか
    \item デバイスドライバとは何か
\end{itemize}


\end{document}