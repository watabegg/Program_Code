\documentclass[a4paper,11pt,dvipdfmx]{jsarticle}

% 数式
\usepackage{amsmath,amsfonts}
\usepackage{bm}

% 画像
\usepackage[dvipdfmx]{graphicx}
\usepackage{framed}

% 図形
\usepackage{tikz}
\usetikzlibrary{shapes.geometric}
\usetikzlibrary{shapes.misc}

% URL
\usepackage{url}

% ソースコード
\usepackage{listings,jlisting,color}
\lstset{
  basicstyle={\ttfamily},
  identifierstyle={\small},
  commentstyle={\small\itshape},
  keywordstyle={\small\bfseries},
  ndkeywordstyle={\small},
  stringstyle={\small\ttfamily},
  frame={tb},
  breaklines=true,
  columns=[l]{fullflexible},
  numbers=left,
  xrightmargin=0zw,
  xleftmargin=3zw,
  numberstyle={\scriptsize},
  stepnumber=1,
  numbersep=1zw,
  lineskip=-0.5ex
}
\renewcommand{\lstlistingname}{ソースコード}

\begin{document}
\definecolor{shadecolor}{gray}{0.70}

\title{機械システム概論 第一回 レポート課題}
\author{21T2166D 渡辺大樹}
\date{\today}
\maketitle

\section*{フィードバック制御の活用事例 - ドローン}
ドローン(無人航空機)は、その多様な活用分野において安定した飛行が求められます。この安定性を確保する上で、外部からの風などの外乱に対応し機体の姿勢を維持するために、フィードバック制御が中心的な役割を担っています。本レポートでは、ドローンの姿勢制御におけるフィードバック制御の活用事例とその概要について述べます。

\section*{ドローンにおけるフィードバック制御の概要}
ドローンの安定飛行の鍵は、機体の傾きや回転(ロール、ピッチ、ヨー)を精密に制御することにあります。この姿勢制御は、フィードバック制御システムによって実現されます。

まず、IMU(慣性計測装置)に代表されるセンサー群が、ドローンの現在の姿勢情報をリアルタイムで検出します\cite{epson}。この実測値と、予め設定された目標姿勢とを比較し、その間のズレ(偏差)を算出します。

次に、この偏差を解消するために、PID制御(比例・積分・微分制御)などのアルゴリズムが用いられます。PID制御は、算出された偏差に基づいて、ドローンの各モーターが出力すべき推力を計算し、適切な制御信号を生成します\cite{tajima}。

この制御信号に従って各モーターの回転数が調整されることで、機体は目標姿勢へと誘導されます。この「センサーによる現状把握 → 偏差計算 → 制御則による指令値生成 → モーターによる姿勢修正」という一連のプロセスが連続的なフィードバックループを形成し、ドローンは風などの外乱の影響を効果的に抑制し、安定した姿勢を保つことができます。

\section*{まとめ}
このように、ドローンの姿勢制御におけるフィードバック制御の活用は、その安定した飛行性能を保証するための根幹技術です。センサーからの情報を基に目標とのズレを補正し続けるこの仕組みにより、ドローンは様々な環境下でその能力を発揮することができます。

\begin{thebibliography}{9}
\bibitem{epson}
エプソン, ``姿勢推定と姿勢制御とは?ドローンやロボットの姿勢安定の仕組み'', \url{https://www.epson.jp/prod/sensing_system/column/attitude-estimation-and-control/}

\bibitem{tajima}
Tajima Robotics, ``ドローンをフィードバック制御するための基本知識を学ぼう!'', \url{https://tajimarobotics.com/drone-feedback-control/}
\end{thebibliography}

\end{document}
